\documentclass{article}
\usepackage[utf8]{inputenc}
\usepackage{amsmath}
\usepackage{amsfonts}
\usepackage{amssymb}
\usepackage[cm]{fullpage}
\usepackage{hyperref}
\usepackage{enumitem}
\usepackage{array}
\usepackage{graphicx}
\usepackage{caption}
\usepackage{subcaption}
\usepackage{float}
\usepackage{moreverb}
\setdescription{labelindent=\parindent}
%%%%%%%%%%%%%%%%%%%%%%%%%%%%%%%%%%%%%%%%%
% Code Snippet
% LaTeX Template
% Version 1.0 (14/2/13)
%
% This template has been downloaded from:
% http://www.LaTeXTemplates.com
%
% Original author:
% Velimir Gayevskiy (vel@latextemplates.com)
%
% License:
% CC BY-NC-SA 3.0 (http://creativecommons.org/licenses/by-nc-sa/3.0/)
%
%%%%%%%%%%%%%%%%%%%%%%%%%%%%%%%%%%%%%%%%%
%----------------------------------------------------------------------------------------

\usepackage{listings} % Required for inserting code snippets
\usepackage[usenames,dvipsnames]{color} % Required for specifying custom colors and referring to colors by name

\definecolor{DarkGreen}{rgb}{0.0,0.4,0.0} % Comment color
\definecolor{highlight}{RGB}{255,251,204} % Code highlight color

\lstdefinestyle{Octave}{ % Define a style for your code snippet, multiple definitions can be made if, for example, you wish to insert multiple code snippets using different programming languages into one document
language=Octave, % Detects keywords, comments, strings, functions, etc for the language specified
%backgroundcolor=\color{highlight}, % Set the background color for the snippet - useful for highlighting
backgroundcolor=\color{white}, % Set the background color for the snippet
basicstyle=\footnotesize\ttfamily, % The default font size and style of the code
breakatwhitespace=false, % If true, only allows line breaks at white space
breaklines=true, % Automatic line breaking (prevents code from protruding outside the box)
captionpos=b, % Sets the caption position: b for bottom; t for top
commentstyle=\usefont{T1}{pcr}{m}{sl}\color{DarkGreen}, % Style of comments within the code - dark green courier font
deletekeywords={}, % If you want to delete any keywords from the current language separate them by commas
%escapeinside={\%}, % This allows you to escape to LaTeX using the character in the bracket
firstnumber=1, % Line numbers begin at line 1
frame=single, % Frame around the code box, value can be: none, leftline, topline, bottomline, lines, single, shadowbox
frameround=tttt, % Rounds the corners of the frame for the top left, top right, bottom left and bottom right positions
keywordstyle=\color{Blue}\bf, % Functions are bold and blue
morekeywords={}, % Add any functions no included by default here separated by commas
numbers=left, % Location of line numbers, can take the values of: none, left, right
numbersep=10pt, % Distance of line numbers from the code box
numberstyle=\tiny\color{Gray}, % Style used for line numbers
rulecolor=\color{black}, % Frame border color
showstringspaces=false, % Don't put marks in string spaces
showtabs=false, % Display tabs in the code as lines
stepnumber=5, % The step distance between line numbers, i.e. how often will lines be numbered
stringstyle=\color{Purple}, % Strings are purple
tabsize=2, % Number of spaces per tab in the code
}
\lstdefinestyle{Python}{ % Define a style for your code snippet, multiple definitions can be made if, for example, you wish to insert multiple code snippets using different programming languages into one document
language=Python, % Detects keywords, comments, strings, functions, etc for the language specified
%backgroundcolor=\color{highlight}, % Set the background color for the snippet - useful for highlighting
backgroundcolor=\color{white}, % Set the background color for the snippet
basicstyle=\footnotesize\ttfamily, % The default font size and style of the code
breakatwhitespace=false, % If true, only allows line breaks at white space
breaklines=true, % Automatic line breaking (prevents code from protruding outside the box)
captionpos=b, % Sets the caption position: b for bottom; t for top
commentstyle=\usefont{T1}{pcr}{m}{sl}\color{DarkGreen}, % Style of comments within the code - dark green courier font
deletekeywords={}, % If you want to delete any keywords from the current language separate them by commas
%escapeinside={\%}, % This allows you to escape to LaTeX using the character in the bracket
firstnumber=1, % Line numbers begin at line 1
frame=single, % Frame around the code box, value can be: none, leftline, topline, bottomline, lines, single, shadowbox
frameround=tttt, % Rounds the corners of the frame for the top left, top right, bottom left and bottom right positions
keywordstyle=\color{Blue}\bf, % Functions are bold and blue
morekeywords={}, % Add any functions no included by default here separated by commas
numbers=left, % Location of line numbers, can take the values of: none, left, right
numbersep=10pt, % Distance of line numbers from the code box
numberstyle=\tiny\color{Gray}, % Style used for line numbers
rulecolor=\color{black}, % Frame border color
showstringspaces=false, % Don't put marks in string spaces
showtabs=false, % Display tabs in the code as lines
stepnumber=5, % The step distance between line numbers, i.e. how often will lines be numbered
stringstyle=\color{Purple}, % Strings are purple
tabsize=2, % Number of spaces per tab in the code
}

% Create a command to cleanly insert a snippet with the style above anywhere in the document
\newcommand{\insertcodeOctave}[2]{\begin{itemize}\item[]\lstinputlisting[caption=#2,label=#1,style=Octave]{#1}\end{itemize}} % The first argument is the script location/filename and the second is a caption for the listing
\newcommand{\insertcodePython}[2]{\begin{itemize}\item[]\lstinputlisting[caption=#2,label=#1,style=Python]{#1}\end{itemize}} % The first argument is the script location/filename and the second is a caption for the listing
% End Code Snippet
%%%%%%%%%%%%%%%%%%%%%%%%%%%%%%%%%%%%%%%%%
% Vector shortcuts
\newcommand*{\X}{\Vec{x}}
\newcommand*{\Y}{\Vec{y}}
\newcommand*{\Z}{\Vec{z}}
\newcommand*{\0}{\Vec{0}}
\newcommand*{\norm}[1]{\lVert#1\rVert_2}
% Greek shortcuts
\newcommand*{\al}{\alpha}
\newcommand*{\be}{\beta}
\newcommand*{\de}{\delta}
\newcommand*{\om}{\omega}
\newcommand*{\ep}{\epsilon}
\begin{document}
\title
{\begin{flushleft}
\large
Patrick Pegus\\
Mini Project 1\\
\today\\
CMPSCI-689\\
Prof. Sridhar Mahadevan
\end{flushleft}}
\author{}
\date{}
\maketitle
\normalsize
\begin{description}
	\item[Theory Question 1] \hfill \\
		To be an inner product, a function must satisfy the following properties:
		\begin{align*}
			\textbf{Non-negativity: }&f(\X,\X) \ge 0 \text{ if $0$, $\X$ must be } \0 \\
			\textbf{Linearity: }&f(\X+\Y,\Z)=f(\X,\Z)+f(\Y,\Z) \\
			\textbf{Scalar multiple: }&f(\al\X,\Y)=\al \cdot f(\X,\Y) \\
			\textbf{Symmetry: }&f(\X,\Y)=f(\Y,\X)
		\end{align*}
%		\begin{description}
%			\item[Non-negativity] \hfill \\
%				This axiom is not satisfied by $\de$. Let $\om_i = \0$. Then $\de(\om_i,\om_i)=\frac{0}{0}$, which is undefined.
%			\item[Linearity] \hfill \\
%				This axiom is not satisfied by $\de$. Let $\norm{\om_i} \ne 1$ or $\norm{\om_j} \ne 1$, for example let $\om_i=[1,2],\om_j=[1,0],\om_k=[0,1]$.
%				Then $\de(\om_i+\om_j,\om_k)\approx 0.7$ and $\de(\om_i,\om_k)+\de(\om_j,\om_k)\approx 0.9$.
%			\item[Scalar multiple] \hfill \\
%				This axiom is not satisfied by $\de$, since it normalizes vectors. For example, let $\al=2$ and $\om_i=[1,0]$.
%				Then $\de(\al\om_i,\om_i)=1$ and $\al\cdot\de(\om_i,\om_i)=2$.
%%				First I prove that $(\al\om_i)^T(\be\om_j) = \al\be\om_i^T\om_j$.
%%				\begin{align*}
%%					(\al\om_i)^T(\be\om_j)
%%					= \sum_l \al\om_{il}\be\om_{jl}
%%					= \al\be\sum_l \om_{il}\om_{jl}
%%					= \al\be\om_i^T\om_j
%%				\end{align*}
%%				Now I prove that $\de$ holds for this axiom.
%%				\begin{align*}
%%					\de(\al\om_i,\om_j) 
%%					= \frac{(\al\om_i)^T\om_j}{\norm{\al\om_i}\norm{\om_j}}
%%					= \frac{(\al\om_i)^T(1\cdot\om_j)}{\sqrt{(\al\om_i)^T(\al\om_i)}\sqrt{\om_j^T\om_j}}
%%					= \frac{\al\om_i^T\om_j}{\sqrt{\al^2\om_i^T\om_i}\sqrt{\om_j^T\om_j}}
%%				\end{align*}
%			\item[Scalar multiple] \hfill \\
%				First I prove that $\om_i^T\om_j = \om_j^T\om_i$.
%				\begin{align*}
%					\om_i^T\om_j
%					&= \sum_l \om_{il}\om_{jl}
%					= \sum_l \om_{jl}\om_{il} \text{, since scalar multiplication is commutative} \\
%					&= \om_j^T\om_i
%				\end{align*}
%				Now I prove that $\de$ holds for this axiom.
%				\begin{align*}
%					\de(\om_i,\om_j) 
%					&= \frac{\om_i^T\om_j}{\norm{\om_i}\norm{\om_j}}
%					= \frac{\om_j^T\om_i}{\norm{\om_i}\norm{\om_j}}
%					= \frac{\om_j^T\om_i}{\norm{\om_j}\norm{\om_i}} \text{, since scalar multiplication is commutative} \\
%					&= \de(\om_j,\om_i) 
%				\end{align*}
%		\end{description}
		The cosine distance $\de$ does not satisfy the scalar multiple property.
		For example, let $\al=2$ and $\om_i^T=[1,0]$.
		Then $\de(\al\om_i,\om_i)=1$ and $\al\cdot\de(\om_i,\om_i)=2$.
	\item[Theory Question 2] \hfill \\
		\begin{table}[H]
			\centering
			\begin{tabular}{|ccccc|rr|}
				\hline
				$a$ & $b$ & $x$ & $y$ & $Dim.$ & $COSADD_y$ & $COSMULT_y$ \\
				\hline
				London & England & Baghdad & Mosul & 50 & 0.648 & 0.845 \\
				London & England & Baghdad & Iraq & 50 & 0.727 & 0.921 \\
				London & England & Baghdad & Mosul & 200 & 0.506 & 0.766 \\
				London & England & Baghdad & Iraq & 200 & 0.574 & 0.835 \\
				\hline
				Cairo & Egypt & Hanoi & Vietnam & 50 & 0.824 & 0.9754 \\
				Cairo & Egypt & Hanoi & Laos & 50 & 0.850 & 0.9746 \\
				\hline
				boy & girl & father & mother & 50 & 0.929 & 0.953 \\
				boy & girl & father & daughter & 50 & 0.926 & 0.956 \\
				\hline
			\end{tabular}
			\caption{
				Given the analogy $a$ is to $b$ as $x$ is to $y$,
				let $COSADD_y = \de(\om_y,\om_x-\om_a+\om_b)$ and
				$COSSMULT_y = \frac{\de(\om_y,\om_b)\de(\om_y,\om_x)}{\de(\om_y,\om_a)+\ep}$.
			}
			\label{tab:comp}
		\end{table}
		As shown in Table ~\ref{tab:comp}, both distance measures yield a greater value for the correct answer, Iraq, regardless of the embedding dimension.
		This disagrees with the results in \cite{levy}, but the computation involves different word embeddings.
		Although I can cherry pick a few examples where COSMULT is more accurate than COSADD, such as the ``Cairo is to Egypt as Hanoi is to Vietnam'', COSADD is generally more accurate on all analogy tasks as seen below.
		Even in the example ``boy is to girl as father is to mother'', where the greater distance, sex, should be relatively reduced by the logarithm, while the smaller distance, age, should be amplified, COSADD outperforms COSMULT.
	\item[Programming Question 3] \hfill \\
		\begin{table}[H]
			\centering
			\begin{tabular}{|cc|rrr|}
				\hline
				Dist. Meas. & Dim. & Sem. & Syn. & Tot. \\
				\hline
				COSADD & 50 & 49.9 & 44.7 & 47.0 \\
				COSMULT & 50 & 36.5 & 27.9 & 31.7 \\
				\hline
				COSADD & 100 & 59.2 & 61.6 & 61.3 \\
				COSMULT & 100 & 55.6 & 41.4 & 47.8 \\
				\hline
				COSADD & 200 & 75.5 & 65.7 & 70.1 \\
				COSMULT & 200 & 56.9 & 40.0 & 47.6 \\
				\hline
			\end{tabular}
			\caption{Comparison of distance measure accuracies on Google word analogies. The various dimension word embeddings were created by GloVe from the Gigaword5 + Wikipedia2014 6B token corpus.}
			\label{tab:google}
		\end{table}
		As shown in Table ~\ref{tab:google}, COSADD significantly outperformed COSMULT in both semantic and syntactic analogy tasks, which is consistent with \cite{glove}.
		Also consistent with \cite{glove}, accuracy generally increases at a decreasing rate with increased word embedding dimensionality.
		Finally, accuracy is higher on semantic tasks with both distance measures.
		This may be caused by GloVe's use of the sum of their model's two output word embedding sets or a larger and more symmetric context window.\cite{glove} 
	\item[Programming Question 4] \hfill \\
		\begin{table}[H]
			\centering
			\begin{tabular}{|cc|r|}
				\hline
				Dist. Meas. & Dim. & Tot. \\
				\hline
				COSADD & 50 & 39.2 \\
				COSMULT & 50 & 21.2 \\
				\hline
				COSADD & 100 & 55.7 \\
				COSMULT & 100 & 38.5 \\
				\hline
				COSADD & 200 & 64.0 \\
				COSMULT & 200 & 41.4 \\
				\hline
			\end{tabular}
			\caption{Comparison of distance measure accuracies on syntactic MSR word analogies. The word embeddings are the same as Table ~\ref{tab:google}.}
			\label{tab:msr}
		\end{table}
		The trends in accuracy between distance measures and across word embedding dimensionality in Table ~\ref{tab:msr} are similar to those seen with the Google word analogies.
		One difference is that the accuracy of COSMULT consistently increases with the dimension of the word embeddings.
\end{description}

\bibliographystyle{abbrv}
\bibliography{report}

\end{document}
